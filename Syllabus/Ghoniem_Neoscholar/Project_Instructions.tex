\documentclass[11pt]{article}

\usepackage[letterpaper,margin=1in,footskip=.5in]{geometry}
\usepackage{kpfonts}
\renewcommand{\familydefault}{\sfdefault}
\normalfont 

%Fonts
\usepackage[scaled]{helvet}
\renewcommand\familydefault{\sfdefault} 
\usepackage[T1]{fontenc}

\usepackage{amsmath,fullpage,color,epsfig,bm,wrapfig,harvard,graphicx,epstopdf,float,subfig,amsthm}
\usepackage[    pdftex,    colorlinks,    hyperindex,    plainpages=false,    bookmarksopen,    bookmarksnumbered  ]{hyperref}
\usepackage[bold]{hhtensor}
\usepackage{nicefrac}

\usepackage[colorlinks]{hyperref} 
\hypersetup{ %Sets up hyperref
    %bookmarks=true,         % show bookmarks bar?
    %unicode=false,          % non-Latin characters in Acrobat?s bookmarks
    %pdftoolbar=true,        % show Acrobat?s toolbar?
    %pdfmenubar=true,        % show Acrobat?s menu?
    %pdffitwindow=false,     % window fit to page when opened
    %pdfstartview={FitH},    % fits the width of the page to the window
    pdftitle={Proposal},    % title
    pdfauthor={Jaime Marian},     % author
    colorlinks=true,       % false: boxed links; true: colored links
    linkcolor=red,          % color of internal links
%    linkcolor=black,          % color of internal links
    citecolor=blue,        % color of links to bibliography
    filecolor=magenta,      % color of file links
    urlcolor=cyan           % color of external links
}
\usepackage{cite}
\usepackage{enumitem}
\usepackage{authblk}
\usepackage{multirow}
\usepackage{etoolbox}
\patchcmd{\thebibliography}{\section*{\refname}}{}{}{}
\addtolength{\textwidth}{.25in}
\addtolength{\topmargin}{.1in}
\addtolength{\textheight}{.25in}
\usepackage{tikz}
\usetikzlibrary{arrows,decorations.pathmorphing,backgrounds,positioning,fit,calc,3d,er,trees}
\usepackage{enumitem} %for itemize left margin
\usepackage{overpic}
\usepackage{float}
\usepackage{cases}
\graphicspath{{./Figures/}}

\title{{\Huge\textcolor{red}{Renewable Energy Research Projects}}}
\author{Nasr Ghoniem\\Distinguished Research Professor\\University of California, Los Angeles (UCLA)\\~ \\\textbf{Course Syllabus}}
\date{\today}

\begin{document}

\maketitle

\tableofcontents
\newpage
\section{Silicon-Based Solar Cell Technology}
\subsection{ Project Statement}
Silicon-based solar cell technology is a cornerstone of modern renewable energy systems, providing an efficient and sustainable means to harness solar energy. This project invites undergraduate students to explore the key principles and applications of silicon photovoltaics. Students will gain insights into the photoelectric effect, bandgap energy, and spectral absorption properties that underlie the conversion of sunlight into electricity. 

The project will also explore the manufacturing processes of silicon solar cells, examining techniques such as wafer production, doping, and anti-reflective coatings. A focus will be placed on evaluating efficiency improvements through advanced designs like passivated emitter rear contact (PERC) cells.

Beyond theoretical knowledge, students will use Python to simulate the Shockley-Queisser efficiency limit for silicon-based cells. The simulator will allow them to calculate and visualize the efficiency as a function of material properties and environmental conditions. The results will include solar spectrum utilization and recommendations for optimizing cell designs.

The project further explores the economic and environmental implications of solar technology, including cost trends, government incentives, and lifecycle emissions. By the end of the project, students will have a comprehensive understanding of silicon-based solar cell technology and its pivotal role in the transition to sustainable energy systems.

\subsection{Physics of Silicon-Based Solar Cells}
Students will dive into the underlying physics that govern silicon solar cell performance, including:
\begin{itemize}
    \item The \textbf{photoelectric effect} and its role in converting sunlight into electricity.
    \item \textbf{Bandgap energy} of silicon and how it influences photon absorption.
    \item \textbf{Spectral properties} of sunlight and how they impact the efficiency of solar cells.
    \item Loss mechanisms, such as thermalization and reflection losses.
\end{itemize}

\textbf{Python Development Idea (Optional - with guidance):}  \\
Students can calculate the theoretical efficiency of a silicon solar cell using the \textbf{Shockley-Queisser limit}. By inputting solar spectral data and the silicon bandgap energy, they can:
\begin{itemize}
    \item Simulate the maximum efficiency of a single-junction silicon solar cell.
    \item Create visualizations of solar spectrum utilization, including absorbed vs. unused energy.
\end{itemize}

\subsection{Manufacturing Process}
Students will examine the manufacturing steps for silicon solar cells, including:
\begin{itemize}
    \item Extraction and purification of silicon from raw materials.
    \item Wafer production: single-crystal silicon (Czochralski process) vs. polycrystalline silicon.
    \item Fabrication steps such as doping, anti-reflective coating, and contact grid application.
    \item Efficiency vs. cost trade-offs in advanced technologies like PERC (Passivated Emitter and Rear Cell).
\end{itemize}

\subsection{Applications}
Students will study the widespread applications of silicon-based solar cells.
\begin{itemize}
    \item Residential and commercial rooftop installations.
    \item Utility-scale solar farms.
    \item Hybrid systems, such as solar battery storage.
    \item Emerging areas, including solar integration into wearables and buildings (e.g., building-integrated photovoltaics, BIPV).
\end{itemize}

\subsection{Economics}
Students will assess the economic aspects of silicon-based solar cell technology.
\begin{itemize}
    \item The cost breakdown of solar cell manufacturing and installation.
    \item Impact of economies of scale and government incentives.
    \item Market trends in the adoption of solar energy globally.
    \item Comparison of silicon-based solar cells to other technologies like thin-film or perovskite solar cells.
\end{itemize}

\subsection*{Environmental Impact}
Students will analyze the environmental implications:
\begin{itemize}
    \item Carbon footprint of silicon solar cell manufacturing.
    \item Comparison of lifecycle emissions with other energy sources.
    \item End-of-life management: challenges in recycling and disposal of silicon solar cells.
    \item Role of solar energy in reducing the reliance on fossil fuels.
\end{itemize}

\subsection{Suggested Python Project}
\textbf{Title:} \textit{Solar Cell Efficiency Simulator}  \\
\textbf{Objective:} Develop a Python program that calculates and visualizes the theoretical efficiency of a silicon solar cell.  \\

\textbf{Features:}
\begin{enumerate}
    \item \textbf{Input Parameters:}
    \begin{itemize}
        \item Solar spectrum data (AM1.5 standard spectrum).
        \item Silicon band gap energy.
        \item Effects of temperature on efficiency.
    \end{itemize}
    \item \textbf{Calculations:}
    \begin{itemize}
        \item Use the Shockley-Queisser model to calculate maximum efficiency.
        \item Analyze the impact of different spectral regions (visible, infrared, etc.) on energy conversion.
    \end{itemize}
    \item \textbf{Visualizations:}
    \begin{itemize}
        \item Plot the solar spectrum with highlighted absorbed and unused regions.
        \item Efficiency as a function of band gap energy for various materials.
    \end{itemize}
    \item \textbf{Output:}
    \begin{itemize}
        \item Report showing the efficiency limits under standard conditions.
        \item Recommendations for optimizing silicon solar cell designs.
    \end{itemize}
\end{enumerate}

\newpage
\subsection{Equations of the Model}

This document describes the equations used in a practical solar cell efficiency model that incorporates key real-world loss mechanisms. The model extends the Shockley-Queisser limit with adjustments for reflection, recombination, resistive, and thermalization losses.
\subsection{Solar Spectrum}
The spectral irradiance $I(E)$ of sunlight is modeled using Planck's law:
\begin{equation}
I(E) = \frac{2 \pi c^2 E^3}{(hc)^3 \left(e^{E / kT_{\text{sun}}} - 1\right)}
\end{equation}
where:
\begin{itemize}
    \item $E$: Photon energy (J),
    \item $c$: Speed of light (m/s),
    \item $h$: Planck's constant (J\,s),
    \item $T_{\text{sun}}$: Sun's temperature (5778 K),
    \item $k$: Boltzmann constant (J/K).
\end{itemize}

\subsection{Absorption of Photons}
Photons with energy $E \geq E_g$ (bandgap energy of the material) are absorbed. The absorbed spectrum is given by:
\begin{equation}
I_{\text{absorbed}}(E) = I(E) \quad \text{for} \quad E \geq E_g
\end{equation}

\subsection{Thermalization Loss}
Photons with an energy higher than the band gap ($E > E_g$) lose their excess energy as heat. The effective photon energy contributing to electricity is:
\begin{equation}
E_{\text{effective}} = \min(E, E_g) \cdot \eta_{\text{thermalization}}
\end{equation}
where $\eta_{\text{thermalization}}$ is the thermalization efficiency (e.g., 80\%).

\subsection{Generated Power}
The generated power $P_{\text{gen}}$ is obtained by integrating the absorbed photon energy over the spectrum:
\begin{equation}
P_{\text{gen}} = (1 - \eta_{\text{reflection}})(1 - \eta_{\text{recombination}})(1 - \eta_{\text{resistance}}) \int_{E_g}^\infty E_{\text{effective}} I_{\text{absorbed}}(E) \, dE
\end{equation}
where:
\begin{itemize}
    \item $\eta_{\text{reflection}}$: Fractional reflection loss (e.g., 10\%),
    \item $\eta_{\text{recombination}}$: Fractional recombination loss (e.g., 20\%),
    \item $\eta_{\text{resistance}}$: Fractional resistive loss (e.g., 5\%).
\end{itemize}

\subsection{Total Incident Power}
The total incident power $P_{\text{total}}$ is the integral of the entire solar spectrum:
\begin{equation}
P_{\text{total}} = \int_0^\infty E \cdot I(E) \, dE
\end{equation}

\subsection{Practical Efficiency}
The practical efficiency $\eta$ is the ratio of generated power to total incident power:
\begin{equation}
\eta = \frac{P_{\text{gen}}}{P_{\text{total}}}
\end{equation}
\subsection{References}
\begin{enumerate}
    \item Shockley, W., \& Queisser, H. J. (1961). "Detailed Balance Limit of Efficiency of p-n Junction Solar Cells." \textit{Journal of Applied Physics}, 32(3), 510--519. \href{https://doi.org/10.1063/1.1736034}{DOI}
    \item Nelson, J. (2003). \textit{The Physics of Solar Cells}. Imperial College Press.
    \item Green, M. A. (1982). "Solar Cell Fill Factors: General Graph and Empirical Expressions." \textit{Solid-State Electronics}, 25(11), 1025--1028.
    \item Knier, G. (2002). "How Do Solar Cells Work?" \textit{NASA Science}. \href{https://science.nasa.gov/astrophysics/focus-areas/how-do-solar-cells-work}{Online Article}
\end{enumerate}



\section*{Project Phases}

\subsection*{Phase 1: Literature Review}
\textbf{Topics to cover:}
\begin{itemize}
    \item Fundamental principles of silicon photovoltaics, including the photoelectric effect and bandgap energy.
    \item Spectral absorption properties of silicon and their impact on energy conversion.
    \item Overview of advanced solar cell designs, including passivated emitter rear contact (PERC) cells.
    \item Environmental and economic impacts of solar cell technologies.
\end{itemize}
\textbf{Deliverable:} A written summary (2--3 pages) highlighting the principles of silicon photovoltaics and key advancements.

\subsection*{Phase 2: Manufacturing Processes}
\textbf{Activities:}
\begin{itemize}
    \item Study the processes involved in silicon solar cell manufacturing, including wafer production, doping, and anti-reflective coatings.
    \item Evaluate manufacturing techniques for their impact on efficiency and cost.
\end{itemize}
\textbf{Deliverable:} A detailed report on manufacturing processes and their role in enhancing solar cell performance.

\subsection*{Phase 3: Simulation and Analysis}
\textbf{Activities:}
\begin{itemize}
    \item Use Python to simulate the Shockley-Queisser efficiency limit for silicon-based cells.
    \item Calculate and visualize efficiency as a function of material properties and environmental conditions.
    \item Analyze results for solar spectrum utilization and provide recommendations for optimizing cell designs.
\end{itemize}
\textbf{Deliverable:} A Python script and accompanying graphs illustrating the efficiency simulation results.

\subsection*{Phase 4: Economic and Environmental Analysis}
\textbf{Activities:}
\begin{itemize}
    \item Investigate cost trends and government incentives for solar technologies.
    \item Perform a lifecycle emissions analysis for silicon-based cells.
\end{itemize}
\textbf{Deliverable:} A comparative analysis report detailing economic feasibility and environmental impacts.

\subsection*{Phase 5: Report and Presentation}
\textbf{Final Report:}
\begin{itemize}
    \item Abstract and introduction.
    \item Literature review summary.
    \item Description of manufacturing processes.
    \item Results of the Python-based simulation.
    \item Economic and environmental analysis.
    \item Conclusions and recommendations for improving solar cell designs.
\end{itemize}
\textbf{Presentation:} A 10-minute presentation summarizing the project.

\section*{Key Learning Outcomes}
\begin{itemize}
    \item Understanding the principles and applications of silicon photovoltaics.
    \item Hands-on experience with Python for simulating solar cell efficiencies.
    \item Insights into manufacturing processes and their impact on performance and cost.
    \item Awareness of the economic and environmental trade-offs in solar technology.
\end{itemize}

\section*{Tools and Resources}
\textbf{Python Libraries:}
\begin{itemize}
    \item \texttt{matplotlib}, \texttt{numpy} for analysis and plotting.
    \item \texttt{pandas} for data handling.
\end{itemize}
\textbf{References:}
\begin{itemize}
    \item Books: \textit{Solar Cell Materials} by Tom Markvart.
    \item Websites: National Renewable Energy Laboratory (NREL), International Renewable Energy Agency (IRENA).
\end{itemize}

\section*{Potential Extensions}
\begin{itemize}
    \item Explore tandem or multi-junction solar cell designs.
    \item Simulate the impact of temperature variations on solar cell efficiency.
    \item Evaluate hybrid systems combining silicon cells with perovskite layers.
\end{itemize}

\section*{Grading Rubric for Solar Project}

\textbf{Total Points: 100}\\
The grading is divided into \textbf{Project Report (70 points)} and \textbf{Final Presentation (30 points)}.

\subsection*{1. Project Report (70 points)}
The report will be assessed based on the following components:

\subsubsection*{A. Literature Review (15 points)}
\begin{itemize}
    \item \textbf{Comprehensiveness (10 points)}:
    \begin{itemize}
        \item Covers key topics: principles of photovoltaics, spectral absorption, and advanced designs.
        \item Properly cites credible references.
    \end{itemize}
    \item \textbf{Clarity and Structure (5 points)}:
    \begin{itemize}
        \item Written clearly and logically, with well-organized sections.
    \end{itemize}
\end{itemize}

\subsubsection*{B. Manufacturing Processes (15 points)}
\begin{itemize}
    \item \textbf{Detail and Accuracy (10 points)}:
    \begin{itemize}
        \item Includes detailed descriptions of key manufacturing steps.
        \item Explains their impact on efficiency and cost.
    \end{itemize}
    \item \textbf{Presentation of Data (5 points)}:
    \begin{itemize}
        \item Data is well-organized using tables, graphs, or illustrations.
    \end{itemize}
\end{itemize}

\subsubsection*{C. Python Code and Simulation (20 points)}
\begin{itemize}
    \item \textbf{Correctness (10 points)}:
    \begin{itemize}
        \item Code executes without errors.
        \item Results align with theoretical predictions of the Shockley-Queisser limit.
    \end{itemize}
    \item \textbf{Visualization and Insights (5 points)}:
    \begin{itemize}
        \item Provides clear and meaningful graphs of simulation results.
    \end{itemize}
    \item \textbf{Documentation and Clarity (5 points)}:
    \begin{itemize}
        \item Code is well-documented with comments explaining logic.
    \end{itemize}
\end{itemize}

\subsubsection*{D. Economic and Environmental Analysis (10 points)}
\begin{itemize}
    \item \textbf{Economic Feasibility (5 points)}:
    \begin{itemize}
        \item Provides detailed calculations for cost trends and incentives.
    \end{itemize}
    \item \textbf{Environmental Impact (5 points)}:
    \begin{itemize}
        \item Addresses lifecycle emissions and sustainability factors.
    \end{itemize}
\end{itemize}

\subsubsection*{E. Report Quality (5 points)}
\begin{itemize}
    \item \textbf{Organization and Flow (3 points)}:
    \begin{itemize}
        \item Sections follow a logical order and are interconnected.
    \end{itemize}
    \item \textbf{Grammar, Style, and Formatting (2 points)}:
    \begin{itemize}
        \item Free of major grammatical errors, formatted consistently.
    \end{itemize}
\end{itemize}

\subsection*{2. Final Presentation (30 points)}
The presentation will be assessed based on the following components:

\subsubsection*{A. Delivery and Communication (10 points)}
\begin{itemize}
    \item \textbf{Clarity and Confidence (5 points)}:
    \begin{itemize}
        \item Speakers demonstrate a clear understanding of the project.
        \item Ideas are communicated confidently and concisely.
    \end{itemize}
    \item \textbf{Audience Engagement (5 points)}:
    \begin{itemize}
        \item Visual aids (slides) are effective and engaging.
        \item Team responds effectively to questions.
    \end{itemize}
\end{itemize}

\subsubsection*{B. Content Coverage (15 points)}
\begin{itemize}
    \item \textbf{Introduction and Objectives (5 points)}:
    \begin{itemize}
        \item Clearly outlines the project objectives and significance.
    \end{itemize}
    \item \textbf{Results and Analysis (5 points)}:
    \begin{itemize}
        \item Key findings, including efficiency simulation results and economic analysis, are presented with graphs or charts.
    \end{itemize}
    \item \textbf{Conclusion and Recommendations (5 points)}:
    \begin{itemize}
        \item Summarizes findings and provides actionable insights.
    \end{itemize}
\end{itemize}

\subsubsection*{C. Time Management (5 points)}
\begin{itemize}
    \item Presentation is delivered within the allotted time (e.g., 10 minutes).
\end{itemize}

\section*{Grading Summary}
\begin{table}[h!]
    \centering
    \begin{tabular}{|l|c|}
        \hline
        \textbf{Category} & \textbf{Points} \\
        \hline
        \textbf{Project Report} & \textbf{70} \\
        Literature Review & 15 \\
        Manufacturing Processes & 15 \\
        Python Code and Simulation & 20 \\
        Economic and Environmental Analysis & 10 \\
        Report Quality & 5 \\
        \hline
        \textbf{Final Presentation} & \textbf{30} \\
        Delivery and Communication & 10 \\
        Content Coverage & 15 \\
        Time Management & 5 \\
        \hline
        \textbf{Total} & \textbf{100} \\
        \hline
    \end{tabular}
    \caption{Grading Rubric for Silicon-Based Solar Cell Project}
\end{table}



\newpage
\section{The Organic Rankine Cycle in Renewable Energy}

\subsection{Project Statement}
The Organic Rankine Cycle (ORC) is a crucial technology in the renewable energy sector, offering a pathway to harness low-grade heat sources for sustainable power generation. This undergraduate project focuses on understanding and simulating the ORC, emphasizing its thermodynamic principles, applications, and environmental benefits. Students will explore the selection of organic working fluids, analyze real-world applications such as geothermal and solar power plants, and evaluate the economic feasibility of ORC systems. 

A significant component of the project involves programming in Python to calculate thermodynamic properties, visualize T-S and H-S diagrams, and determine efficiency and power flows using tools like CoolProp. In addition, students will examine the environmental impacts of ORCs, discussing their role in reducing greenhouse gas emissions and using waste heat effectively.

The project integrates theoretical knowledge with practical skills, allowing students to analyze and model energy systems while considering economic and environmental factors. By the end of this project, participants will have a comprehensive understanding of ORC technology and its pivotal role in the advancement of renewable energy solutions. 


\subsection{Thermodynamics and Selection of Organic Fluids}
This section explores the thermodynamic principles of the ORC and the criteria for selecting working fluids. Key considerations include:
\begin{itemize}
    \item Critical temperature and pressure.
    \item Thermal stability.
    \item Environmental and safety properties.
    \item Efficiency impact.
\end{itemize}
Diagrams of typical organic fluid T-S and P-H plots will be discussed.

\subsection{Applications of Organic Rankine Cycle}
The ORC is utilized in various renewable energy applications:
\begin{itemize}
    \item Geothermal power plants.
    \item Solar thermal systems.
    \item Biomass power plants.
    \item Industrial waste heat recovery.
\end{itemize}
Real-world examples and case studies will be provided.

\subsection{Thermodynamic Equations for Calculating Efficiency and Power Flows}
The ORC efficiency and power flows are calculated using fundamental thermodynamic equations:
\begin{align}
    \eta &= \frac{W_{net}}{Q_{in}} \\
    W_{net} &= W_{turbine} - W_{pump} \\
    Q_{in} &= \dot{m}(h_3 - h_2)
\end{align}
The derivation of these equations will be detailed along with assumptions and boundary conditions.

\subsection{Python Project for Efficiency and Power Flows in an ORC}
This section outlines a Python-based project to model and simulate the ORC. Students will:
\begin{enumerate}
    \item Input parameters: working fluid, heat source temperature, and pressure.
    \item Calculate thermodynamic properties using \texttt{CoolProp}.
    \item Plot T-S and H-S diagrams for the cycle.
    \item Compute the efficiency and net power output.
\end{enumerate}
An example Python script will be included.

\subsection{Economics of Typical ORCs}
The economic feasibility of ORCs is examined, considering:
\begin{itemize}
    \item Capital costs.
    \item Operating and maintenance costs.
    \item Payback period.
    \item Economic incentives and subsidies.
\end{itemize}
Case studies from commercial ORC installations will be analyzed.

\subsection{Environmental Impact of ORC}
The environmental benefits of ORCs include:
\begin{itemize}
    \item Reduction in greenhouse gas emissions.
    \item Utilization of renewable and waste heat sources.
    \item Minimal environmental footprint of organic fluids.
\end{itemize}
The potential environmental hazards of organic fluids will also be discussed.

\subsection{Conclusions}
This section summarizes the role of the ORC in renewable energy, highlighting its advantages, challenges, and future prospects.

\section*{Project Phases}

\subsection*{Phase 1: Literature Review}
\textbf{Topics to cover:}
\begin{itemize}
    \item Fundamental principles of the Organic Rankine Cycle, including thermodynamic properties and processes.
    \item Selection criteria for organic working fluids (e.g., thermal stability, efficiency, environmental impact).
    \item Real-world applications in geothermal, solar power, and waste heat recovery.
    \item Environmental and economic impacts of ORC technology.
\end{itemize}
\textbf{Deliverable:} A written summary (2--3 pages) highlighting the principles, applications, and benefits of ORC technology.

\subsection*{Phase 2: Simulation and Thermodynamic Analysis}
\textbf{Activities:}
\begin{itemize}
    \item Use Python and CoolProp to calculate thermodynamic properties of organic fluids.
    \item Plot T-S and H-S diagrams to visualize the ORC cycle.
    \item Determine efficiency and power flows based on input conditions and working fluid properties.
\end{itemize}
\textbf{Deliverable:} A Python script and accompanying graphs illustrating the ORC cycle and efficiency calculations.

\subsection*{Phase 3: Economic Feasibility Study}
\textbf{Activities:}
\begin{itemize}
    \item Perform a cost-benefit analysis of ORC systems, considering capital costs, maintenance, and operating expenses.
    \item Evaluate economic feasibility based on payback periods and energy cost savings.
\end{itemize}
\textbf{Deliverable:} A report detailing the economic analysis of ORC systems.

\subsection*{Phase 4: Environmental Impact Analysis}
\textbf{Activities:}
\begin{itemize}
    \item Assess greenhouse gas emission reductions through ORC implementation.
    \item Evaluate the role of ORCs in effective waste heat utilization.
\end{itemize}
\textbf{Deliverable:} A comparative report on the environmental benefits and challenges of ORC technology.

\subsection*{Phase 5: Report and Presentation}
\textbf{Final Report:}
\begin{itemize}
    \item Abstract and introduction.
    \item Literature review summary.
    \item Results of thermodynamic simulations and efficiency calculations.
    \item Economic and environmental analysis.
    \item Conclusions and recommendations for optimizing ORC systems.
\end{itemize}
\textbf{Presentation:} A 10-minute presentation summarizing the project.

\section*{Key Learning Outcomes}
\begin{itemize}
    \item Understanding the principles and applications of ORC technology.
    \item Hands-on experience with Python for thermodynamic modeling and visualization.
    \item Insights into the economic and environmental trade-offs of ORC systems.
    \item Development of skills in analyzing and designing renewable energy systems.
\end{itemize}

\section*{Tools and Resources}
\textbf{Python Libraries:}
\begin{itemize}
    \item \texttt{matplotlib}, \texttt{numpy} for analysis and plotting.
    \item \texttt{CoolProp} for thermodynamic property calculations.
    \item \texttt{pandas} for data handling.
\end{itemize}
\textbf{References:}
\begin{itemize}
    \item Books: \textit{Organic Rankine Cycle Technology for Energy Recovery} by Ennio Macchi.
    \item Websites: National Renewable Energy Laboratory (NREL), International Renewable Energy Agency (IRENA).
\end{itemize}

\section*{Potential Extensions}
\begin{itemize}
    \item Simulate hybrid ORC systems combining geothermal and solar heat sources.
    \item Explore the impact of superheating and regenerative cycles on ORC efficiency.
    \item Assess the feasibility of novel working fluids with low global warming potential (GWP).
\end{itemize}

\section*{Grading Rubric for ORC Project}

\textbf{Total Points: 100}\\
The grading is divided into \textbf{Project Report (70 points)} and \textbf{Final Presentation (30 points)}.

\subsection*{1. Project Report (70 points)}
The report will be assessed based on the following components:

\subsubsection*{A. Literature Review (15 points)}
\begin{itemize}
    \item \textbf{Comprehensiveness (10 points)}:
    \begin{itemize}
        \item Covers key topics: ORC principles, fluid selection, and real-world applications.
        \item Properly cites credible references.
    \end{itemize}
    \item \textbf{Clarity and Structure (5 points)}:
    \begin{itemize}
        \item Written clearly and logically, with well-organized sections.
    \end{itemize}
\end{itemize}

\subsubsection*{B. Thermodynamic Simulation (20 points)}
\begin{itemize}
    \item \textbf{Correctness (10 points)}:
    \begin{itemize}
        \item Code executes without errors and produces accurate results.
        \item Results align with thermodynamic principles.
    \end{itemize}
    \item \textbf{Visualization and Insights (5 points)}:
    \begin{itemize}
        \item Provides clear and meaningful graphs of T-S and H-S diagrams.
    \end{itemize}
    \item \textbf{Documentation and Clarity (5 points)}:
    \begin{itemize}
        \item Code is well-documented with comments explaining logic.
    \end{itemize}
\end{itemize}

\subsubsection*{C. Economic Feasibility (15 points)}
\begin{itemize}
    \item \textbf{Detail and Accuracy (10 points)}:
    \begin{itemize}
        \item Includes detailed cost-benefit analysis.
        \item Explains economic feasibility based on payback period and energy cost savings.
    \end{itemize}
    \item \textbf{Clarity (5 points)}:
    \begin{itemize}
        \item Results are presented clearly, using tables or graphs where appropriate.
    \end{itemize}
\end{itemize}

\subsubsection*{D. Environmental Analysis (10 points)}
\begin{itemize}
    \item \textbf{Impact Assessment (5 points)}:
    \begin{itemize}
        \item Effectively evaluates greenhouse gas emission reductions.
    \end{itemize}
    \item \textbf{Sustainability Insights (5 points)}:
    \begin{itemize}
        \item Discusses the role of ORCs in sustainable energy systems.
    \end{itemize}
\end{itemize}

\subsubsection*{E. Report Quality (10 points)}
\begin{itemize}
    \item \textbf{Organization and Flow (5 points)}:
    \begin{itemize}
        \item Sections follow a logical order and are interconnected.
    \end{itemize}
    \item \textbf{Grammar, Style, and Formatting (5 points)}:
    \begin{itemize}
        \item Free of major grammatical errors, formatted consistently.
    \end{itemize}
\end{itemize}

\subsection*{2. Final Presentation (30 points)}
The presentation will be assessed based on the following components:

\subsubsection*{A. Delivery and Communication (10 points)}
\begin{itemize}
    \item \textbf{Clarity and Confidence (5 points)}:
    \begin{itemize}
        \item Speakers demonstrate a clear understanding of the project.
        \item Ideas are communicated confidently and concisely.
    \end{itemize}
    \item \textbf{Audience Engagement (5 points)}:
    \begin{itemize}
        \item Visual aids (slides) are effective and engaging.
        \item Team responds effectively to questions.
    \end{itemize}
\end{itemize}

\subsubsection*{B. Content Coverage (15 points)}
\begin{itemize}
    \item \textbf{Introduction and Objectives (5 points)}:
    \begin{itemize}
        \item Clearly outlines the project objectives and significance.
    \end{itemize}
    \item \textbf{Results and Analysis (5 points)}:
    \begin{itemize}
        \item Key findings, including thermodynamic simulations and economic analysis, are presented with graphs or charts.
    \end{itemize}
    \item \textbf{Conclusion and Recommendations (5 points)}:
    \begin{itemize}
        \item Summarizes findings and provides actionable insights.
    \end{itemize}
\end{itemize}

\subsubsection*{C. Time Management (5 points)}
\begin{itemize}
    \item Presentation is delivered within the allotted time (e.g., 10 minutes).
\end{itemize}
\begin{table}[h!]
    \centering
    \begin{tabular}{|l|c|}
        \hline
        \textbf{Category} & \textbf{Points} \\
        \hline
        \textbf{Project Report} & \textbf{70} \\
        Literature Review & 15 \\
        Thermodynamic Simulation & 20 \\
        Economic Feasibility & 15 \\
        Environmental Analysis & 10 \\
        Report Quality & 10 \\
        \hline
        \textbf{Final Presentation} & \textbf{30} \\
        Delivery and Communication & 10 \\
        Content Coverage & 15 \\
        Time Management & 5 \\
        \hline
        \textbf{Total} & \textbf{100} \\
        \hline
    \end{tabular}
    \caption{Grading Rubric for Organic Rankine Cycle (ORC) Project}
\end{table}

\newpage


\section{Design and Analysis of a Small-Scale Hydroelectric Power Plant Using Python}
The goal of this project is to provide students with a comprehensive understanding of hydroelectric power generation. The project will involve:
\begin{itemize}
    \item Review of the literature on hydro power systems and their environmental and economic impacts.
    \item Data collection and analysis of site-specific conditions for hydro plant design (e.g., head, flow rate, and turbine selection).
    \item Development or use of Python code for plant design and performance analysis.
    \item An evaluation of the feasibility and sustainability of the proposed design.
    \item Preparation of a detailed project report documenting the findings, methodology, and conclusions.
\end{itemize}

\subsection{Project Phases}

\subsection*{Phase 1: Literature Review}
\textbf{Topics to cover:}
\begin{itemize}
    \item Types of hydro power plants (run-of-river, storage, pumped storage).
    \item Key components: dams, turbines, generators, and penstocks.
    \item Environmental and social impacts of hydro projects.
    \item Current advancements in turbine technology and small-scale hydro systems.
\end{itemize}
\textbf{Deliverable:} A written summary (2--3 pages) highlighting the importance of hydro power, key design considerations, and its role in renewable energy.

\subsection*{Phase 2: Site Analysis}
\textbf{Activities:}
\begin{itemize}
    \item Select or assume a hypothetical or real site for the hydro plant.
    \item Research or assume site-specific parameters like river flow rates (Q), available head (H), and seasonal variations.
    \item Evaluate potential turbine types based on head and flow rate ranges.
\end{itemize}
\textbf{Deliverable:} A dataset summarizing head, flow, and seasonal variations, along with the rationale for selecting a particular turbine type.

\subsection*{Phase 3: Design and Simulation}
\textbf{Activities:}
\begin{itemize}
    \item Use Python code to:
    \begin{itemize}
        \item Calculate the power output for different flow rates and head values.
        \item Perform economic analysis, including capital cost, annual revenue, and payback period.
        \item Simulate seasonal variability in power output.
    \end{itemize}
    \item Extend the code to include:
    \begin{itemize}
        \item Additional parameters like penstock friction losses or turbine efficiency curves.
        \item Optimization of power output and cost.
    \end{itemize}
\end{itemize}
\textbf{Deliverable:} A Python script that models the power plant and generates useful visualizations (e.g., power vs. flow rate, seasonal output, and cost breakdown).

\subsection*{Phase 4: Comparative Analysis}
\textbf{Activities:}
\begin{itemize}
    \item Compare the design with a real-world small-scale hydro plant or a case study.
    \item Evaluate how variations in assumptions (e.g., turbine efficiency or head) impact power generation and economic viability.
\end{itemize}
\textbf{Deliverable:} A 1--2 page comparison report with graphs and key insights.

\subsection*{Phase 5: Report and Presentation}
\textbf{Final Report:}
\begin{itemize}
    \item Abstract and introduction.
    \item Literature review summary.
    \item Methodology for site selection and design.
    \item Results of power and economic analysis, with visualizations.
    \item Conclusions and recommendations for improving plant performance.
\end{itemize}
\textbf{Presentation:} A 10-minute presentation summarizing the project.

\subsection{Key Learning Outcomes}
\begin{itemize}
    \item Understanding the fundamentals of hydroelectric power generation.
    \item Hands-on experience with Python for engineering design and analysis.
    \item Insights into the trade-offs between cost, efficiency, and environmental impacts in energy projects.
    \item Development of technical reporting and communication skills.
\end{itemize}

\subsection{Tools and Resources}
\textbf{Python Libraries:}
\begin{itemize}
    \item \texttt{matplotlib}, \texttt{numpy} for analysis and plotting.
    \item Optional: \texttt{pandas} for data handling.
\end{itemize}
\textbf{Data Sources:}
\begin{itemize}
    \item Open datasets for river flow rates (e.g., USGS streamflow data).
    \item Case studies from organizations like the International Hydropower Association (IHA).
\end{itemize}
\textbf{References:}
\begin{itemize}
    \item Books: \textit{Hydropower Engineering Handbook} by C. S. Gupta.
    \item Websites: International Renewable Energy Agency (IRENA), IHA.
\end{itemize}

\subsection{Potential Extensions}
\begin{itemize}
    \item Incorporate environmental impact analysis using Python.
    \item Simulate pumped storage systems with energy storage cycles.
    \item Evaluate hybrid systems combining hydro with solar or wind.
\end{itemize}

\section*{Grading Rubric for Hydropower Project}

\textbf{Total Points: 100}\\
The grading is divided into \textbf{Project Report (70 points)} and \textbf{Final Presentation (30 points)}.

\section*{1. Project Report (70 points)}
The report will be assessed based on the following components:

\subsection*{A. Literature Review (15 points)}
\begin{itemize}
    \item \textbf{Comprehensiveness (10 points)}:
    \begin{itemize}
        \item Covers key topics: types of hydro power plants, turbines, environmental/economic impacts.
        \item Properly cites credible references.
    \end{itemize}
    \item \textbf{Clarity and Structure (5 points)}:
    \begin{itemize}
        \item Written clearly and logically, with well-organized sections.
    \end{itemize}
\end{itemize}

\subsection*{B. Site Analysis (15 points)}
\begin{itemize}
    \item \textbf{Data Quality (10 points)}:
    \begin{itemize}
        \item Includes relevant site-specific parameters (head, flow rate, seasonal variations).
        \item Provides rationale for turbine selection with reference to technical specifications.
    \end{itemize}
    \item \textbf{Presentation of Data (5 points)}:
    \begin{itemize}
        \item Data is well-organized using tables, graphs, or charts.
    \end{itemize}
\end{itemize}

\subsection*{C. Python Code and Simulation (20 points)}
\begin{itemize}
    \item \textbf{Correctness (10 points)}:
    \begin{itemize}
        \item Code executes without errors.
        \item Results align with input parameters and hydro power calculations.
    \end{itemize}
    \item \textbf{Extension and Innovation (5 points)}:
    \begin{itemize}
        \item Includes extensions like efficiency curves, friction losses, or other enhancements.
    \end{itemize}
    \item \textbf{Documentation and Clarity (5 points)}:
    \begin{itemize}
        \item Code is well-documented with comments explaining logic.
    \end{itemize}
\end{itemize}

\subsection*{D. Economic and Environmental Analysis (10 points)}
\begin{itemize}
    \item \textbf{Economic Viability (5 points)}:
    \begin{itemize}
        \item Provides detailed calculations for capital cost, revenue, and payback period.
    \end{itemize}
    \item \textbf{Environmental Impact (5 points)}:
    \begin{itemize}
        \item Addresses potential environmental trade-offs or sustainability factors.
    \end{itemize}
\end{itemize}

\subsection*{E. Comparative Analysis (5 points)}
\begin{itemize}
    \item \textbf{Depth of Comparison (3 points)}:
    \begin{itemize}
        \item Effectively compares the design with a real-world case study or benchmarks.
    \end{itemize}
    \item \textbf{Insights and Recommendations (2 points)}:
    \begin{itemize}
        \item Provides meaningful conclusions based on the comparison.
    \end{itemize}
\end{itemize}

\subsection*{F. Report Quality (5 points)}
\begin{itemize}
    \item \textbf{Organization and Flow (3 points)}:
    \begin{itemize}
        \item Sections follow a logical order and are interconnected.
    \end{itemize}
    \item \textbf{Grammar, Style, and Formatting (2 points)}:
    \begin{itemize}
        \item Free of major grammatical errors, formatted consistently.
    \end{itemize}
\end{itemize}

\section*{2. Final Presentation (30 points)}
The presentation will be assessed based on the following components:

\subsection*{A. Delivery and Communication (10 points)}
\begin{itemize}
    \item \textbf{Clarity and Confidence (5 points)}:
    \begin{itemize}
        \item Speakers demonstrate a clear understanding of the project.
        \item Ideas are communicated confidently and concisely.
    \end{itemize}
    \item \textbf{Audience Engagement (5 points)}:
    \begin{itemize}
        \item Visual aids (slides) are effective and engaging.
        \item Team responds effectively to questions.
    \end{itemize}
\end{itemize}

\subsection*{B. Content Coverage (15 points)}
\begin{itemize}
    \item \textbf{Introduction and Objectives (5 points)}:
    \begin{itemize}
        \item Clearly outlines the project objectives and significance.
    \end{itemize}
    \item \textbf{Results and Analysis (5 points)}:
    \begin{itemize}
        \item Key findings, including power output, economic analysis, and seasonal variability, are presented with graphs or charts.
    \end{itemize}
    \item \textbf{Conclusion and Recommendations (5 points)}:
    \begin{itemize}
        \item Summarizes findings and provides actionable insights.
    \end{itemize}
\end{itemize}

\subsection*{C. Time Management (5 points)}
\begin{itemize}
    \item Presentation is delivered within the allotted time (e.g., 10 minutes).
\end{itemize}

\section*{Grading Summary}
\begin{table}[h!]
    \centering
    \begin{tabular}{|l|c|}
        \hline
        \textbf{Category} & \textbf{Points} \\
        \hline
        \textbf{Project Report} & \textbf{70} \\
        Literature Review & 15 \\
        Site Analysis & 15 \\
        Python Code and Simulation & 20 \\
        Economic and Environmental Analysis & 10 \\
        Comparative Analysis & 5 \\
        Report Quality & 5 \\
        \hline
        \textbf{Final Presentation} & \textbf{30} \\
        Delivery and Communication & 10 \\
        Content Coverage & 15 \\
        Time Management & 5 \\
        \hline
        \textbf{Total} & \textbf{100} \\
        \hline
    \end{tabular}
    \caption{Grading Rubric for Hydropower Project}
\end{table}


\newpage
\appendix
\section{Appendix: Python code for a single solar cell efficiency simulator}

\begin{lstlisting}[style=custompython, caption=Python Code for Solar Cell Efficiency Simulator]
import numpy as np
import matplotlib.pyplot as plt
from scipy.constants import h, c, k

# Constants
q = 1.602e-19  # Elementary charge (C)
hc = h * c  # Planck's constant times the speed of light (J*m)
T_sun = 5778  # Sun's temperature (K)
kT_sun = k * T_sun

# Reasonable Loss Parameters
reflection_loss = 0.1  # 10% of sunlight is reflected
recombination_loss = 0.2  # 20% of generated carriers are lost
resistive_loss = 0.05  # 5% of power is lost due to resistance
thermalization_efficiency = 0.8  # 80% efficiency for photons above bandgap

# Functions
def solar_spectrum(E_photon):
    spectrum = (2 * np.pi * c**2 * E_photon**3) / ((hc)**3 * (np.exp(E_photon / kT_sun) - 1))
    return spectrum

def calculate_efficiency(E_g, E_photon, spectrum):
    absorbed = E_photon >= E_g
    absorbed_spectrum = spectrum[absorbed]
    absorbed_energy = E_photon[absorbed]
    effective_energy = np.minimum(absorbed_energy, E_g) * thermalization_efficiency
    power_generated = np.trapz(effective_energy * absorbed_spectrum, x=E_photon[absorbed])
    power_generated *= (1 - reflection_loss) * (1 - recombination_loss) * (1 - resistive_loss)
    total_power = np.trapz(E_photon * spectrum, x=E_photon)
    efficiency = power_generated / total_power
    return efficiency

E_photon = np.linspace(0.1, 4, 1000) * q
spectrum = solar_spectrum(E_photon)
E_g = 1.1 * q
efficiency = calculate_efficiency(E_g, E_photon, spectrum)
print(f"Practical Efficiency with Losses: {efficiency * 100:.2f}%")

plt.figure(figsize=(10, 6))
plt.plot(E_photon / q, spectrum, label="Solar Spectrum (AM1.5)")
plt.axvline(E_g / q, color="r", linestyle="--", label=f"Bandgap Energy (E_g = {E_g / q:.2f} eV)")
plt.fill_between(E_photon / q, spectrum, where=(E_photon >= E_g), alpha=0.3, label="Absorbed Spectrum")
plt.title("Solar Spectrum and Absorption with Losses")
plt.xlabel("Photon Energy (eV)")
plt.ylabel("Spectral Intensity")
plt.legend()
plt.grid()
plt.show()
\end{lstlisting}
\newpage
\section{Appendix: Python code for a multi-junction solar cell efficiency simulator}

\begin{lstlisting}[style=custompython, caption=Python Code for Solar Cell Efficiency Simulator]
import numpy as np
import matplotlib.pyplot as plt
from scipy.constants import h, c, k

# Constants
q = 1.602e-19  # Elementary charge (C)
hc = h * c  # Planck's constant times the speed of light (J*m)

# Solar spectrum (AM1.5)
def solar_spectrum(E_photon):
    T_sun = 5778  # Sun's temperature (K)
    kT_sun = k * T_sun
    spectrum = (2 * np.pi * c**2 * E_photon**3) / ((hc)**3 * (np.exp(E_photon / kT_sun) - 1))
    return spectrum

# Absorption calculation
def absorption(alpha, d, E_photon):
    return 1 - np.exp(-alpha * d)

# Layer-by-layer absorption and power contribution
def layer_efficiency(E_g, alpha, d, E_photon, transmitted_spectrum):
    absorbed = absorption(alpha, d, E_photon)
    usable_photons = (E_photon >= E_g)
    effective_energy = np.minimum(E_photon, E_g) * absorbed * usable_photons
    layer_power = np.trapz(effective_energy * transmitted_spectrum, x=E_photon)
    
    # Reduce transmitted spectrum for subsequent layers
    remaining_spectrum = transmitted_spectrum * (1 - absorbed)
    return layer_power, remaining_spectrum

# Total efficiency function
def total_efficiency(layers, E_photon, spectrum):
    transmitted_spectrum = spectrum  # Start with full spectrum
    total_power = 0
    incident_power = np.trapz(E_photon * spectrum, x=E_photon)  # Total incident power
    for layer in layers:
        E_g, alpha, d = layer
        layer_power, transmitted_spectrum = layer_efficiency(E_g, alpha, d, E_photon, transmitted_spectrum)
        total_power += layer_power
    return total_power / incident_power

# Efficiency with loss mechanisms
def practical_efficiency(layers, E_photon, spectrum, losses):
    transmitted_spectrum = spectrum  # Start with full spectrum
    total_power = 0
    incident_power = np.trapz(E_photon * spectrum, x=E_photon)  # Total incident power
    for layer in layers:
        E_g, alpha, d = layer
        layer_power, transmitted_spectrum = layer_efficiency(E_g, alpha, d, E_photon, transmitted_spectrum)
        
        # Apply loss mechanisms
        layer_power *= (1 - losses['reflection']) * (1 - losses['thermalization']) * (1 - losses['recombination'])
        total_power += layer_power
    return total_power / incident_power

# Material and layer properties
layers = [
    # Example: [Bandgap (J), Absorption coefficient (1/m), Thickness (m)]
    [1.85 * q, 1e7, 1e-6],  # Layer 1: GaInP2
    [1.424 * q, 1e7, 2e-6],  # Layer 2: GaAs
    [0.66 * q, 1e6, 3e-6],  # Layer 3: Ge
]

# Photon energy range (in Joules)
E_photon = np.linspace(0.1, 4, 1000) * q
spectrum = solar_spectrum(E_photon)

# Ideal efficiency
ideal_efficiency = total_efficiency(layers, E_photon, spectrum)
print(f"Ideal Efficiency: {ideal_efficiency * 100:.2f}%")

# Loss mechanisms
losses = {
    'reflection': 0.1,  # 10% reflection loss
    'thermalization': 0.2,  # 20% thermalization loss
    'recombination': 0.15,  # 15% recombination loss
}

# Practical efficiency
practical_efficiency_value = practical_efficiency(layers, E_photon, spectrum, losses)
print(f"Practical Efficiency: {practical_efficiency_value * 100:.2f}%")

# Plot individual graphs for each layer
for i, layer in enumerate(layers):
    E_g, alpha, d = layer
    absorption_curve = absorption(alpha, d, E_photon)
    absorbed_spectrum = absorption_curve * spectrum
    
    # Shade only the portion above the bandgap
    mask = E_photon >= E_g
    absorbed_spectrum_above_bandgap = absorbed_spectrum * mask

    plt.figure(figsize=(10, 6))
    plt.plot(E_photon / q, spectrum, label="Solar Spectrum (AM1.5)", color="black")
    plt.fill_between(E_photon / q, absorbed_spectrum_above_bandgap, color=f"C{i}", alpha=0.3, label=f"Layer {i+1} Absorbed")
    plt.axvline(E_g / q, color=f"C{i}", linestyle="--", label=f"Layer {i+1} Bandgap {E_g / q:.2f} eV")

    plt.title(f"Absorption for Layer {i+1}")
    plt.xlabel("Photon Energy (eV)")
    plt.ylabel("Spectral Intensity (arbitrary units)")
    plt.legend()
    plt.grid()
    plt.show()

\end{lstlisting}
\newpage
\section{Python code for thermodynamic properties of R134a}
\begin{lstlisting}[style=custompython, caption=Python code for thermodynamic properties of R134a]
    from CoolProp.CoolProp import PropsSI

def r134a_table_examples():
    """
    Examples for calculating R134a properties using CoolProp, covering:
    1. General R134a properties.
    2. Superheated state: Given temperature and pressure, find all properties.
    3. Saturation region: Given entropy and pressure, find the refrigerant quality (x).
    4. Saturation region: Given entropy and pressure, find properties.
    5. Liquid state: Given pressure and x=0, find all properties.
    6. Liquid state: Given pressure and entropy, find all properties.
    """

    # (1) General R134a Properties: Given P & T
    print("(1) General R134a Properties")
    pressure = 500000  # Pa (5 bar)
    temperature = 300  # K (27°C)
    print(f"Pressure: {pressure / 1000:.2f} kPa, Temperature: {temperature - 273.15:.2f} °C")
    print(f"Specific Enthalpy: {PropsSI('H', 'P', pressure, 'T', temperature, 'R134a') / 1000:.2f} kJ/kg")
    print(f"Specific Entropy: {PropsSI('S', 'P', pressure, 'T', temperature, 'R134a'):.4f} J/kg.K\n")

    # (2) Superheated State: Given T & P, find all properties
    print("(2) Superheated State")
    pressure = 600000  # Pa (6 bar)
    temperature = 40 + 273.15  # K (40°C)
    print(f"Pressure: {pressure / 1000:.2f} kPa, Temperature: {temperature - 273.15:.2f} °C")
    print(f"Specific Enthalpy: {PropsSI('H', 'P', pressure, 'T', temperature, 'R134a') / 1000:.2f} kJ/kg")
    print(f"Specific Entropy: {PropsSI('S', 'P', pressure, 'T', temperature, 'R134a'):.4f} J/kg.K\n")

    # (3) Saturation Region: Given S & P, find quality (x)
    print("(3) Saturation Region: Find Quality (x)")
    pressure = 200000  # Pa (2 bar)
    entropy_liquid = PropsSI('S', 'P', pressure, 'Q', 0, 'R134a')  # Saturated liquid entropy
    entropy_vapor = PropsSI('S', 'P', pressure, 'Q', 1, 'R134a')  # Saturated vapor entropy
    entropy = entropy_liquid + 0.5 * (entropy_vapor - entropy_liquid)  # Ensure quality < 1 (e.g., 0.5)
    quality = (entropy - entropy_liquid) / (entropy_vapor - entropy_liquid)  # Quality calculation
    print(f"Pressure: {pressure / 1000:.2f} kPa, Entropy: {entropy / 1000:.2f} kJ/kg.K")
    print(f"Quality (x): {quality:.4f}\n")

    # (4) Saturation Region: Given S & P, find properties
    print("(4) Saturation Region: Find Properties")
    enthalpy_liquid = PropsSI('H', 'P', pressure, 'Q', 0, 'R134a')  # Saturated liquid enthalpy
    enthalpy_vapor = PropsSI('H', 'P', pressure, 'Q', 1, 'R134a')  # Saturated vapor enthalpy
    enthalpy = enthalpy_liquid + quality * (enthalpy_vapor - enthalpy_liquid)  # Enthalpy for quality
    print(f"Pressure: {pressure / 1000:.2f} kPa")
    print(f"Entropy: {entropy / 1000:.2f} kJ/kg.K, Quality (x): {quality:.2f}")
    print(f"Enthalpy: {enthalpy / 1000:.2f} kJ/kg\n")

    # (5) Liquid State: Given P & x=0, find all properties
    print("(5) Liquid State: Find All Properties")
    pressure = 400000  # Pa (4 bar)
    print(f"Pressure: {pressure / 1000:.2f} kPa")
    print(f"Saturated Liquid Enthalpy: {PropsSI('H', 'P', pressure, 'Q', 0, 'R134a') / 1000:.2f} kJ/kg")
    print(f"Saturated Liquid Density: {PropsSI('D', 'P', pressure, 'Q', 0, 'R134a'):.2f} kg/m^3\n")

    # (6) Liquid State: Given P & S, find all properties
    print("(6) Liquid State: Find Properties")
    pressure = 300000  # Pa (3 bar)
    entropy = 1.0  # kJ/kg.K
    temperature = PropsSI('T', 'P', pressure, 'S', entropy * 1000, 'R134a')  # Temperature from S & P
    print(f"Pressure: {pressure / 1000:.2f} kPa, Entropy: {entropy:.2f} kJ/kg.K")
    print(f"Temperature: {temperature - 273.15:.2f} °C")
    print(f"Enthalpy: {PropsSI('H', 'P', pressure, 'S', entropy * 1000, 'R134a') / 1000:.2f} kJ/kg")
    print(f"Density: {PropsSI('D', 'P', pressure, 'S', entropy * 1000, 'R134a'):.2f} kg/m^3\n")

# Call the function to run all examples
r134a_table_examples()
\end{lstlisting}


\newpage
\section{Appendix: Python code for the Ideal Rankine Cycle}

\begin{lstlisting}[style=custompython, caption=Python code for the Ideal Rankine Cycle]
from CoolProp.CoolProp import PropsSI
import matplotlib.pyplot as plt
import numpy as np

def rankine_cycle_analysis():
    """
    Perform calculations for the ideal Rankine cycle, including:
    1. Specific work of turbine and pump.
    2. Heat added (qin) and rejected (qout).
    3. Net work (wnet) and thermal efficiency (eta).
    Generate T-S and H-S diagrams with saturation dome and cycle curves.
    """

    # Define state points (pressures in Pa, temperatures in K)
    P1 = 100000  # Condenser pressure (Pa)
    P2 = 5000000  # Boiler pressure (Pa)
    T3 = 773.15  # Turbine inlet temperature (K)
    Pcrit = 22e6
    dome_p2 = Pcrit
    dome_p1 = 1e4
    # State 1: Saturated liquid at P1
    h1 = PropsSI('H', 'P', P1, 'Q', 0, 'Water')  # Enthalpy (J/kg)
    s1 = PropsSI('S', 'P', P1, 'Q', 0, 'Water')  # Entropy (J/kg.K)

    # State 2: Compressed liquid (isentropic process)
    s2 = s1  # Isentropic process
    h2 = PropsSI('H', 'P', P2, 'S', s2, 'Water')
    T2 = PropsSI('T', 'P', P2, 'S', s2, 'Water')

    # State 2_prime: Saturated liquid at P2
    h2_prime = PropsSI('H', 'P', P2, 'Q', 0, 'Water')
    s2_prime = PropsSI('S', 'P', P2, 'Q', 0, 'Water')

    # State 3: Superheated vapor
    h3 = PropsSI('H', 'P', P2, 'Q', 1,  'Water')  # Enthalpy (J/kg)
    s3 = PropsSI('S', 'P', P2, 'Q', 1, 'Water')

    # State 4: Saturated mixture (isentropic expansion)
    s4 = s3  # Isentropic process
    h4 = PropsSI('H', 'P', P1, 'S', s4, 'Water')

    # Work and heat calculations
    wturb = h3 - h4  # Work done by the turbine (J/kg)
    wpump = h2 - h1  # Work done by the pump (J/kg)
    qin = h3 - h2  # Heat added in the boiler (J/kg)
    qout = h1 - h4  # Heat rejected in the condenser (J/kg)

    # Net work and thermal efficiency
    wnet = wturb - wpump  # Net work (J/kg)
    eta = wnet / qin  # Thermal efficiency

    # Display results
    print("Rankine Cycle Analysis")
    print(f"State 1: h1 = {h1 / 1000:.2f} kJ/kg, s1 = {s1 / 1000:.4f} kJ/kg.K")
    print(f"State 2: h2 = {h2 / 1000:.2f} kJ/kg, s2 = {s2 / 1000:.4f} kJ/kg.K, T2 = {T2 - 273.15:.2f} °C")
    print(f"State 2_prime: h2_prime = {h2_prime / 1000:.2f} kJ/kg, s2_prime = {s2_prime / 1000:.4f} kJ/kg.K")
    print(f"State 3: h3 = {h3 / 1000:.2f} kJ/kg, s3 = {s3 / 1000:.4f} kJ/kg.K")
    print(f"State 4: h4 = {h4 / 1000:.2f} kJ/kg, s4 = {s4 / 1000:.4f} kJ/kg.K")
    print(f"Turbine Work (wturb): {wturb / 1000:.2f} kJ/kg")
    print(f"Pump Work (wpump): {wpump / 1000:.2f} kJ/kg")
    print(f"Heat Added (qin): {qin / 1000:.2f} kJ/kg")
    print(f"Heat Rejected (qout): {qout / 1000:.2f} kJ/kg")
    print(f"Net Work (wnet): {wnet / 1000:.2f} kJ/kg")
    print(f"Thermal Efficiency (eta): {eta * 100:.2f}%")

    # Generate T-S Diagram
    s_sat_liquid = []
    t_sat_liquid = []
    s_sat_vapor = []
    t_sat_vapor = []

    for p in np.logspace(np.log10(dome_p1), np.log10(dome_p2), 200):
        s_sat_liquid.append(PropsSI('S', 'P', p, 'Q', 0, 'Water'))
        t_sat_liquid.append(PropsSI('T', 'P', p, 'Q', 0, 'Water'))
        s_sat_vapor.append(PropsSI('S', 'P', p, 'Q', 1, 'Water'))
        t_sat_vapor.append(PropsSI('T', 'P', p, 'Q', 1, 'Water'))

    plt.figure()
    # Saturation dome
    plt.plot([s / 1000 for s in s_sat_liquid], [t - 273.15 for t in t_sat_liquid], label="Saturation Liquid", linestyle="-", linewidth=1)
    plt.plot([s / 1000 for s in s_sat_vapor], [t - 273.15 for t in t_sat_vapor], label="Saturation Vapor", linestyle="-", linewidth=1)

    # Cycle curve
    cycle_s = [s1, s2, s2_prime, s3, s4, s1]
    cycle_t = [PropsSI('T', 'P', P1, 'Q', 0, 'Water') - 273.15, T2 - 273.15, PropsSI('T', 'P', P2, 'Q', 0, 'Water') - 273.15,
               PropsSI('T', 'P', P2, 'Q', 1, 'Water') - 273.15, PropsSI('T', 'P', P1, 'Q', 0, 'Water') - 273.15, PropsSI('T', 'P', P1, 'Q', 0, 'Water') - 273.15]
    plt.plot([s / 1000 for s in cycle_s], cycle_t, label="Cycle", linewidth=2, color="red")

    # Label states
    for i, (s, t) in enumerate(zip(cycle_s, cycle_t), start=1):
        plt.text(s / 1000, t, f"State {i}", fontsize=8)

    plt.xlabel("Entropy (kJ/kg.K)")
    plt.ylabel("Temperature (°C)")
    plt.title("Rankine Cycle T-S Diagram")
    ylim = 500
    plt.legend()
    plt.grid()
    plt.show()

    # Generate H-S Diagram (Mollier Diagram)
    h_sat_liquid = []
    s_sat_liquid = []
    h_sat_vapor = []
    s_sat_vapor = []

    for p in np.logspace(np.log10(dome_p1), np.log10(dome_p2), 500):
        h_sat_liquid.append(PropsSI('H', 'P', p, 'Q', 0, 'Water'))
        s_sat_liquid.append(PropsSI('S', 'P', p, 'Q', 0, 'Water'))
        h_sat_vapor.append(PropsSI('H', 'P', p, 'Q', 1, 'Water'))
        s_sat_vapor.append(PropsSI('S', 'P', p, 'Q', 1, 'Water'))

    plt.figure()
    # Saturation dome
    plt.plot([s / 1000 for s in s_sat_liquid], [h / 1000 for h in h_sat_liquid], label="Saturation Liquid", linestyle="-", linewidth=1)
    plt.plot([s / 1000 for s in s_sat_vapor], [h / 1000 for h in h_sat_vapor], label="Saturation Vapor", linestyle="-", linewidth=1)

    # Cycle curve
    cycle_h = [h1, h2, h2_prime, h3, h4, h1]
    cycle_s = [s1, s2, s2_prime, s3, s4, s1]
    plt.plot([s / 1000 for s in cycle_s], [h / 1000 for h in cycle_h], label="Cycle", linewidth=2, color="red")

    # Label states
    for i, (s, h) in enumerate(zip(cycle_s, cycle_h), start=1):
        plt.text(s / 1000, h / 1000, f"State {i}", fontsize=8)

    plt.xlabel("Entropy (kJ/kg.K)")
    plt.ylabel("Enthalpy (kJ/kg)")
    plt.title("Rankine Cycle H-S Diagram (Mollier Diagram)")
    plt.legend()
    plt.grid()
    plt.show()

rankine_cycle_analysis()
\end{lstlisting}
\newpage
\section{Appendix: Design of a Hydro Power Plant}
\begin{lstlisting}[style=custompython, caption=Python Code for Design of a Hydro Power Plant]
    import numpy as np
import matplotlib.pyplot as plt

# Constants
g = 9.81  # Gravitational acceleration (m/s^2)
rho = 1000  # Water density (kg/m^3)

# User Inputs
head = 50  # Net head in meters
flow_rate = 20  # Flow rate in cubic meters per second (m^3/s)
efficiency = 0.9  # Efficiency of the turbine (fraction, e.g., 90%)

# Function to calculate power
def calculate_power(flow_rate, head, efficiency):
    """
    Calculate the power output of a hydroelectric power plant.
    :param flow_rate: Flow rate (m^3/s)
    :param head: Head (m)
    :param efficiency: Turbine efficiency (fraction)
    :return: Power (kW)
    """
    power = efficiency * rho * g * flow_rate * head
    return power / 1000  # Convert to kW

# Turbine Selection
turbine_types = {
    "Pelton": {"head_range": (50, 1000), "flow_range": (1, 10)},
    "Francis": {"head_range": (10, 300), "flow_range": (5, 50)},
    "Kaplan": {"head_range": (2, 20), "flow_range": (10, 100)},
}

def select_turbine(head, flow_rate):
    """
    Select a suitable turbine based on head and flow rate.
    """
    for turbine, specs in turbine_types.items():
        if specs["head_range"][0] <= head <= specs["head_range"][1] and specs["flow_range"][0] <= flow_rate <= specs["flow_range"][1]:
            return turbine
    return "No suitable turbine found"

# Economic Analysis
def economic_analysis(power_output, cost_per_kw=1500, electricity_price=0.1):
    """
    Calculate capital cost, annual revenue, and payback period.
    :param power_output: Power output (kW)
    :param cost_per_kw: Capital cost per kW ($)
    :param electricity_price: Price per kWh ($/kWh)
    :return: Capital cost, annual revenue, payback period (years)
    """
    capital_cost = power_output * cost_per_kw
    annual_revenue = power_output * 8760 * electricity_price  # 8760 hours/year
    payback_period = capital_cost / annual_revenue
    return capital_cost, annual_revenue, payback_period

# Seasonal Variability
def seasonal_variability(head, flow_rate, months):
    """
    Simulate power output over months with variable flow rates.
    :param head: Head (m)
    :param flow_rate: Average flow rate (m^3/s)
    :param months: List of monthly flow rate factors (e.g., [0.8, 1.2, ...])
    :return: List of monthly power outputs (kW)
    """
    monthly_powers = []
    for factor in months:
        monthly_flow = flow_rate * factor
        monthly_power = calculate_power(monthly_flow, head, efficiency)
        monthly_powers.append(monthly_power)
    return monthly_powers

# Main Execution
power_output = calculate_power(flow_rate, head, efficiency)
selected_turbine = select_turbine(head, flow_rate)

# Economic Analysis
capital_cost, annual_revenue, payback_period = economic_analysis(power_output)

# Seasonal Variability Simulation
monthly_factors = [0.7, 0.8, 1.0, 1.2, 1.1, 0.9, 0.8, 0.7, 0.9, 1.0, 1.1, 1.2]
monthly_powers = seasonal_variability(head, flow_rate, monthly_factors)

# Display Results
print(f"Power Output: {power_output:.2f} kW")
print(f"Selected Turbine: {selected_turbine}")
print(f"Capital Cost: ${capital_cost:,.2f}")
print(f"Annual Revenue: ${annual_revenue:,.2f}")
print(f"Payback Period: {payback_period:.2f} years")

# Plot Seasonal Power Output
plt.figure()
months = np.arange(1, 13)
plt.bar(months, monthly_powers, color="skyblue", label="Monthly Power Output")
plt.xlabel("Month")
plt.ylabel("Power Output (kW)")
plt.title("Seasonal Variability in Power Output")
plt.xticks(months)
plt.grid(axis="y")
plt.legend()
plt.show()

# Plot Efficiency Curve
def plot_efficiency_curve():
    flow_rates = np.linspace(5, 50, 100)  # Vary flow rate between 5 and 50 m^3/s
    power_outputs = [calculate_power(flow, head, efficiency) for flow in flow_rates]
    efficiency_curve = np.linspace(0.7, 0.95, len(flow_rates))  # Example efficiency curve
    adjusted_power = [calculate_power(flow, head, eff) for flow, eff in zip(flow_rates, efficiency_curve)]

    plt.figure()
    plt.plot(flow_rates, power_outputs, label="Ideal Power Output", linestyle="--")
    plt.plot(flow_rates, adjusted_power, label="Adjusted Power Output")
    plt.xlabel("Flow Rate (m^3/s)")
    plt.ylabel("Power Output (kW)")
    plt.title("Efficiency Impact on Power Output")
    plt.legend()
    plt.grid()
    plt.show()

plot_efficiency_curve()
\end{lstlisting}

\end{document}